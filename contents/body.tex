%% contents/body.tex
\section{系统架构}
% 架构图与模块说明
\begin{frame}{整体架构设计}
	\begin{columns}
		\column{.4\textwidth}
		\begin{figure}
			\centering
			% 请替换为您的类图或架构图
			\includegraphics[width=.95\textwidth]{contents/figure/architecture_placeholder.png}
			\caption{游戏类继承体系}
			\label{fig:architecture}
		\end{figure}
		
		\column{.6\textwidth}
		\begin{itemize}
			\item \textbf{实体系统 (Entity System)}:
			\begin{itemize}
				\item \texttt{BaseEntity}: 所有游戏对象的基类,处理坐标、精灵渲染。
				\item \texttt{Building}: 派生出 \texttt{DefenseTower} (防御塔), \texttt{ResourceMine} (资源矿)。
				\item \texttt{Troop}: 派生出 \texttt{Barbarian}, \texttt{Archer} 等兵种,包含状态机。
			\end{itemize}
			\item \textbf{管理层 (Managers)}:
			\begin{itemize}
				\item \texttt{GameManager}: 单例模式,管理全局游戏状态(战斗/非战斗)。
				\item \texttt{MapManager}: 负责瓦片地图 (Tiled Map) 的加载与坐标转换。
			\end{itemize}
		\end{itemize}
	\end{columns}
\end{frame}

\section{关键技术实现}
% AI 与 寻路逻辑
\begin{frame}{战斗AI与寻路算法}
	\begin{columns}
		\column{.5\textwidth}
		\begin{block}{混合式寻路策略}
			为保证百余单位同屏的性能,采用了分层寻路:
			\begin{enumerate}
				\item \textbf{全局寻路}: 使用 \textbf{A* 算法} 计算从出发点到目标建筑的最短路径,规避城墙。
				\item \textbf{局部避障}: 单位移动时检测相邻格子碰撞,动态调整微小位移。
			\end{enumerate}
		\end{block}
		
		\column{.5\textwidth}
		\begin{figure}
			\centering
			% 请替换为游戏战斗截图或寻路演示图
			\includegraphics[width=.9\textwidth]{contents/figure/battle_ai_placeholder.png}
			\caption{A* 寻路路径可视化}
		\end{figure}
	\end{columns}
	
	\vspace{0.5em}
	\textbf{目标优先级判定函数}:
	$$ Priority(target) = \frac{W_{type} \times TypeFactor}{Distance(me, target)} $$
	其中 $W_{type}$ 为兵种偏好权重(如巨人优先攻击防御塔,$W_{def}=10$)。
\end{frame}

% C++ 特性展示(复用原模版代码排版)
\begin{frame}{C++ 高级特性应用}
	\begin{itemize}
		\item \textbf{STL 容器与算法}:
		\small 使用 \texttt{std::vector<Troop*>} 管理兵种池,利用 \texttt{std::sort} 根据 Y 轴坐标动态调整渲染层级(遮挡关系)。
		\item \textbf{多态与虚函数}:
		\small \texttt{Attack()} 接口在基类中定义,不同兵种(近战/远程)重写该方法实现不同的攻击逻辑(近战判定碰撞盒,远程发射抛物线子弹)。
		\item \textbf{内存管理}:
		\small 结合 Cocos2d-x 的 \texttt{Retain/Release} 机制与 C++11 \texttt{std::shared\_ptr},杜绝内存泄漏。
	\end{itemize}
	
	\begin{block}{代码规范}
		项目严格遵循 Google C++ Style Guide,代码复用率高,核心逻辑通过 \texttt{git} 进行版本控制,Commit 记录清晰。
	\end{block}
\end{frame}